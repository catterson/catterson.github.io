% Options for packages loaded elsewhere
% Options for packages loaded elsewhere
\PassOptionsToPackage{unicode}{hyperref}
\PassOptionsToPackage{hyphens}{url}
\PassOptionsToPackage{dvipsnames,svgnames,x11names}{xcolor}
%
\documentclass[
  letterpaper,
  DIV=11,
  numbers=noendperiod,
  oneside]{scrartcl}
\usepackage{xcolor}
\usepackage[left=1in,marginparwidth=2.0666666666667in,textwidth=4.1333333333333in,marginparsep=0.3in]{geometry}
\usepackage{amsmath,amssymb}
\setcounter{secnumdepth}{-\maxdimen} % remove section numbering
\usepackage{iftex}
\ifPDFTeX
  \usepackage[T1]{fontenc}
  \usepackage[utf8]{inputenc}
  \usepackage{textcomp} % provide euro and other symbols
\else % if luatex or xetex
  \usepackage{unicode-math} % this also loads fontspec
  \defaultfontfeatures{Scale=MatchLowercase}
  \defaultfontfeatures[\rmfamily]{Ligatures=TeX,Scale=1}
\fi
\usepackage{lmodern}
\ifPDFTeX\else
  % xetex/luatex font selection
\fi
% Use upquote if available, for straight quotes in verbatim environments
\IfFileExists{upquote.sty}{\usepackage{upquote}}{}
\IfFileExists{microtype.sty}{% use microtype if available
  \usepackage[]{microtype}
  \UseMicrotypeSet[protrusion]{basicmath} % disable protrusion for tt fonts
}{}
\makeatletter
\@ifundefined{KOMAClassName}{% if non-KOMA class
  \IfFileExists{parskip.sty}{%
    \usepackage{parskip}
  }{% else
    \setlength{\parindent}{0pt}
    \setlength{\parskip}{6pt plus 2pt minus 1pt}}
}{% if KOMA class
  \KOMAoptions{parskip=half}}
\makeatother
% Make \paragraph and \subparagraph free-standing
\makeatletter
\ifx\paragraph\undefined\else
  \let\oldparagraph\paragraph
  \renewcommand{\paragraph}{
    \@ifstar
      \xxxParagraphStar
      \xxxParagraphNoStar
  }
  \newcommand{\xxxParagraphStar}[1]{\oldparagraph*{#1}\mbox{}}
  \newcommand{\xxxParagraphNoStar}[1]{\oldparagraph{#1}\mbox{}}
\fi
\ifx\subparagraph\undefined\else
  \let\oldsubparagraph\subparagraph
  \renewcommand{\subparagraph}{
    \@ifstar
      \xxxSubParagraphStar
      \xxxSubParagraphNoStar
  }
  \newcommand{\xxxSubParagraphStar}[1]{\oldsubparagraph*{#1}\mbox{}}
  \newcommand{\xxxSubParagraphNoStar}[1]{\oldsubparagraph{#1}\mbox{}}
\fi
\makeatother

\usepackage{color}
\usepackage{fancyvrb}
\newcommand{\VerbBar}{|}
\newcommand{\VERB}{\Verb[commandchars=\\\{\}]}
\DefineVerbatimEnvironment{Highlighting}{Verbatim}{commandchars=\\\{\}}
% Add ',fontsize=\small' for more characters per line
\usepackage{framed}
\definecolor{shadecolor}{RGB}{241,243,245}
\newenvironment{Shaded}{\begin{snugshade}}{\end{snugshade}}
\newcommand{\AlertTok}[1]{\textcolor[rgb]{0.68,0.00,0.00}{#1}}
\newcommand{\AnnotationTok}[1]{\textcolor[rgb]{0.37,0.37,0.37}{#1}}
\newcommand{\AttributeTok}[1]{\textcolor[rgb]{0.40,0.45,0.13}{#1}}
\newcommand{\BaseNTok}[1]{\textcolor[rgb]{0.68,0.00,0.00}{#1}}
\newcommand{\BuiltInTok}[1]{\textcolor[rgb]{0.00,0.23,0.31}{#1}}
\newcommand{\CharTok}[1]{\textcolor[rgb]{0.13,0.47,0.30}{#1}}
\newcommand{\CommentTok}[1]{\textcolor[rgb]{0.37,0.37,0.37}{#1}}
\newcommand{\CommentVarTok}[1]{\textcolor[rgb]{0.37,0.37,0.37}{\textit{#1}}}
\newcommand{\ConstantTok}[1]{\textcolor[rgb]{0.56,0.35,0.01}{#1}}
\newcommand{\ControlFlowTok}[1]{\textcolor[rgb]{0.00,0.23,0.31}{\textbf{#1}}}
\newcommand{\DataTypeTok}[1]{\textcolor[rgb]{0.68,0.00,0.00}{#1}}
\newcommand{\DecValTok}[1]{\textcolor[rgb]{0.68,0.00,0.00}{#1}}
\newcommand{\DocumentationTok}[1]{\textcolor[rgb]{0.37,0.37,0.37}{\textit{#1}}}
\newcommand{\ErrorTok}[1]{\textcolor[rgb]{0.68,0.00,0.00}{#1}}
\newcommand{\ExtensionTok}[1]{\textcolor[rgb]{0.00,0.23,0.31}{#1}}
\newcommand{\FloatTok}[1]{\textcolor[rgb]{0.68,0.00,0.00}{#1}}
\newcommand{\FunctionTok}[1]{\textcolor[rgb]{0.28,0.35,0.67}{#1}}
\newcommand{\ImportTok}[1]{\textcolor[rgb]{0.00,0.46,0.62}{#1}}
\newcommand{\InformationTok}[1]{\textcolor[rgb]{0.37,0.37,0.37}{#1}}
\newcommand{\KeywordTok}[1]{\textcolor[rgb]{0.00,0.23,0.31}{\textbf{#1}}}
\newcommand{\NormalTok}[1]{\textcolor[rgb]{0.00,0.23,0.31}{#1}}
\newcommand{\OperatorTok}[1]{\textcolor[rgb]{0.37,0.37,0.37}{#1}}
\newcommand{\OtherTok}[1]{\textcolor[rgb]{0.00,0.23,0.31}{#1}}
\newcommand{\PreprocessorTok}[1]{\textcolor[rgb]{0.68,0.00,0.00}{#1}}
\newcommand{\RegionMarkerTok}[1]{\textcolor[rgb]{0.00,0.23,0.31}{#1}}
\newcommand{\SpecialCharTok}[1]{\textcolor[rgb]{0.37,0.37,0.37}{#1}}
\newcommand{\SpecialStringTok}[1]{\textcolor[rgb]{0.13,0.47,0.30}{#1}}
\newcommand{\StringTok}[1]{\textcolor[rgb]{0.13,0.47,0.30}{#1}}
\newcommand{\VariableTok}[1]{\textcolor[rgb]{0.07,0.07,0.07}{#1}}
\newcommand{\VerbatimStringTok}[1]{\textcolor[rgb]{0.13,0.47,0.30}{#1}}
\newcommand{\WarningTok}[1]{\textcolor[rgb]{0.37,0.37,0.37}{\textit{#1}}}

\usepackage{longtable,booktabs,array}
\usepackage{calc} % for calculating minipage widths
% Correct order of tables after \paragraph or \subparagraph
\usepackage{etoolbox}
\makeatletter
\patchcmd\longtable{\par}{\if@noskipsec\mbox{}\fi\par}{}{}
\makeatother
% Allow footnotes in longtable head/foot
\IfFileExists{footnotehyper.sty}{\usepackage{footnotehyper}}{\usepackage{footnote}}
\makesavenoteenv{longtable}
\usepackage{graphicx}
\makeatletter
\newsavebox\pandoc@box
\newcommand*\pandocbounded[1]{% scales image to fit in text height/width
  \sbox\pandoc@box{#1}%
  \Gscale@div\@tempa{\textheight}{\dimexpr\ht\pandoc@box+\dp\pandoc@box\relax}%
  \Gscale@div\@tempb{\linewidth}{\wd\pandoc@box}%
  \ifdim\@tempb\p@<\@tempa\p@\let\@tempa\@tempb\fi% select the smaller of both
  \ifdim\@tempa\p@<\p@\scalebox{\@tempa}{\usebox\pandoc@box}%
  \else\usebox{\pandoc@box}%
  \fi%
}
% Set default figure placement to htbp
\def\fps@figure{htbp}
\makeatother

\ifLuaTeX
  \usepackage{luacolor}
  \usepackage[soul]{lua-ul}
\else
  \usepackage{soul}
\fi




\setlength{\emergencystretch}{3em} % prevent overfull lines

\providecommand{\tightlist}{%
  \setlength{\itemsep}{0pt}\setlength{\parskip}{0pt}}



 


\KOMAoption{captions}{tableheading}
\makeatletter
\@ifpackageloaded{caption}{}{\usepackage{caption}}
\AtBeginDocument{%
\ifdefined\contentsname
  \renewcommand*\contentsname{Table of contents}
\else
  \newcommand\contentsname{Table of contents}
\fi
\ifdefined\listfigurename
  \renewcommand*\listfigurename{List of Figures}
\else
  \newcommand\listfigurename{List of Figures}
\fi
\ifdefined\listtablename
  \renewcommand*\listtablename{List of Tables}
\else
  \newcommand\listtablename{List of Tables}
\fi
\ifdefined\figurename
  \renewcommand*\figurename{Figure}
\else
  \newcommand\figurename{Figure}
\fi
\ifdefined\tablename
  \renewcommand*\tablename{Table}
\else
  \newcommand\tablename{Table}
\fi
}
\@ifpackageloaded{float}{}{\usepackage{float}}
\floatstyle{ruled}
\@ifundefined{c@chapter}{\newfloat{codelisting}{h}{lop}}{\newfloat{codelisting}{h}{lop}[chapter]}
\floatname{codelisting}{Listing}
\newcommand*\listoflistings{\listof{codelisting}{List of Listings}}
\makeatother
\makeatletter
\makeatother
\makeatletter
\@ifpackageloaded{caption}{}{\usepackage{caption}}
\@ifpackageloaded{subcaption}{}{\usepackage{subcaption}}
\makeatother
\makeatletter
\@ifpackageloaded{sidenotes}{}{\usepackage{sidenotes}}
\@ifpackageloaded{marginnote}{}{\usepackage{marginnote}}
\makeatother
\usepackage{bookmark}
\IfFileExists{xurl.sty}{\usepackage{xurl}}{} % add URL line breaks if available
\urlstyle{same}
\hypersetup{
  pdftitle={Chapter 1 \textbar{} Why Statistics?},
  colorlinks=true,
  linkcolor={blue},
  filecolor={Maroon},
  citecolor={Blue},
  urlcolor={Blue},
  pdfcreator={LaTeX via pandoc}}


\title{Chapter 1 \textbar{} Why Statistics?}
\author{}
\date{}
\begin{document}
\maketitle


\section{Hello Again!}\label{hello-again}

\begin{itemize}
\item
  \textbf{these notes on bCourses, or :}
  \href{https://catterson.github.io/calstats/calstatsSP26.html}{catterson.github.io/calstats/calstatsSP26.html}
\item
  find these notes, then
  \href{https://docs.google.com/forms/d/e/1FAIpQLSekU91xhyXiYxxOsWbSGuJ3pAmPOazq9L_24hkcQpej-n0bBg/viewform?usp=header}{\textbf{click
  on this link to check-in}}
\end{itemize}

\begin{center}
\pandocbounded{\includegraphics[keepaspectratio]{images/clipboard-792871999.png}}
\end{center}

\subsection{Announcements}\label{announcements}

\begin{itemize}
\tightlist
\item
  \textbf{Section Swap :} post on bCourses your \& swap buddy info (now
  open.)
\item
  \textbf{Office Hours :} Fridays at 9:00 AM on Zoom.
\item
  Commuinicate course conflicts with GSI (okay to miss lecture; watch
  recording \& do check-in.)
\item
  \textbf{Next Week :} Section, Lab 1 (\emph{Partial Key Posted}),
  Chapter 2 Quiz, Lecture, Repeat.
\end{itemize}

\subsection{Agenda}\label{agenda}

\begin{itemize}
\tightlist
\item
  2:10 - 2:20 : Check-In \& Announcements
\item
  2:20 - 3:20 : Variables
\item
  3:20 - 3:30 : Break \#1
\item
  3:30 - 4:00 : Linear Models and Predictions
\item
  4:00 - 4:05 : Break \#2
\item
  4:05 - 4:30 : So you're interested in going to graduate school?
\item
  4:30 - 5:00 : DISCUSSION : Project Questions
\end{itemize}

\section{Part 1 : Variables}\label{part-1-variables}

\subsection{RECAP : Defining Research
Questions}\label{recap-defining-research-questions}

\subsubsection{Student Examples from Milestone
\#1}\label{student-examples-from-milestone-1}

\begin{itemize}
\tightlist
\item
  What's the variable at the focus of the following questions?
\item
  What does between vs.~within-person variation look like? Which one is
  more interesting to you?
\item
  Are you interested in the AFFECT, BEHAVIOR, or COGNITIVE part of this
  variable?
\item
  How would you measure this variable with numbers? Do you think this is
  possible?
\end{itemize}

\subsubsection{Developing Research Questions from
Real-Life}\label{developing-research-questions-from-real-life}

\begin{itemize}
\tightlist
\item
  What are some real-life events that we want to better understand or
  change?
\item
  What are some research questions that we might develop based on
  real-life events?
\item
  How can we think about these questions using the ABCs of Psychology?
\item
  Is this a question about between-person or within-person variation?
\item
  How would you measure this variable with numbers? Do you think this is
  possible?
\end{itemize}

\subsection{From Research Questions to Variables in
R}\label{from-research-questions-to-variables-in-r}

\subsubsection{Numeric Variables in R}\label{numeric-variables-in-r}

\pandocbounded{\includegraphics[keepaspectratio]{images/clipboard-812337385.png}}

\subsubsection{Example In R : Creating Numeric
Variables}\label{example-in-r-creating-numeric-variables}

\begin{Shaded}
\begin{Highlighting}[]
\NormalTok{num.var }\OtherTok{\textless{}{-}} \FunctionTok{c}\NormalTok{(}\DecValTok{6}\NormalTok{,}\DecValTok{7}\NormalTok{,}\DecValTok{6}\NormalTok{,}\DecValTok{7}\NormalTok{,}\DecValTok{6}\NormalTok{,}\DecValTok{7}\NormalTok{,}\DecValTok{6}\NormalTok{,}\DecValTok{7}\NormalTok{)}
\FunctionTok{hist}\NormalTok{(num.var)}
\end{Highlighting}
\end{Shaded}

\pandocbounded{\includegraphics[keepaspectratio]{1L_WhyStats_files/figure-pdf/unnamed-chunk-1-1.pdf}}

\subsubsection{String Variables in R}\label{string-variables-in-r}

\pandocbounded{\includegraphics[keepaspectratio]{images/clipboard-2893888681.png}}

\subsubsection{Example in R : Creating Non-Numeric
Variables}\label{example-in-r-creating-non-numeric-variables}

\begin{Shaded}
\begin{Highlighting}[]
\NormalTok{cat.var }\OtherTok{\textless{}{-}} \FunctionTok{c}\NormalTok{(}\StringTok{"string"}\NormalTok{, }\StringTok{"string"}\NormalTok{, }\StringTok{"string"}\NormalTok{, }\StringTok{"yarn"}\NormalTok{)}
\NormalTok{cat.var }\OtherTok{\textless{}{-}} \FunctionTok{as.factor}\NormalTok{(cat.var)}
\FunctionTok{plot}\NormalTok{(cat.var)}
\end{Highlighting}
\end{Shaded}

\pandocbounded{\includegraphics[keepaspectratio]{1L_WhyStats_files/figure-pdf/unnamed-chunk-2-1.pdf}}

\subsubsection{\texorpdfstring{\textbf{ACTIVTY :} Work on Lab 1,
Question
1.}{ACTIVTY : Work on Lab 1, Question 1.}}\label{activty-work-on-lab-1-question-1.}

\begin{enumerate}
\def\labelenumi{\arabic{enumi}.}
\tightlist
\item
  \textbf{Access Lab 1 (in the Week notes).} Copy and paste the
  questions into a Word or Google Document.
\item
  \textbf{Open RStudio and a new RScript.} Type some math into the
  Rscript, and send it to the console. THAT'S RIGHT. YOU ARE A
  PROGRAMMER. FRIEND OF COMPUTERS. CODE WIZARD.
\item
  \textbf{Work on Lab 1, Question 1 :} Define two variables in R : one
  numeric, and one string. \emph{Note : you can use the data that we
  collected in class, or collect your own data (make sure each variable
  has at least ten data points).} ``Print'' each variable in R, and
  paste the output as a screenshot. Then, graph the numeric variable as
  a ``histogram'' and the string variable as a ``plot''. Below each
  graph, describe what you observe about the individuals in the dataset
  for each variable.
\end{enumerate}

\subsection{Break Time: Meet Back at}\label{break-time-meet-back-at}

\url{https://www.youtube.com/watch?v=NUnJc82ptd4}

\section{From Variables to Predictions (and
Beyond)}\label{from-variables-to-predictions-and-beyond}

\subsection{Prediction \& Power in Real
Life}\label{prediction-power-in-real-life}

EXAMPLE : what will your bedtime be tonight?

\begin{itemize}
\tightlist
\item
  \textbf{Knowledge (what information did you use to make the
  prediction)?}
\item
  \textbf{Power (ways your predictions influence future behaviors)}
\item
  \textbf{Was The Prediction Valid?}
\end{itemize}

\subsubsection{\texorpdfstring{\textbf{Work on Lab Question
2.}}{Work on Lab Question 2.}}\label{work-on-lab-question-2.}

What's a prediction about people (or the world) that you made today?
What information did you use to make this prediction? How did (or could)
you use this prediction to influence outcomes? Were you valid in your
predictions? Finally, write a linear model that defines the prediction
(and information that you used) as a formula (e.g., DV \textasciitilde{}
IV1 + IV2 + \ldots{} + error).

\subsection{Scientific Predictions}\label{scientific-predictions}

Psychological scientists seek to better understand variation, in order
to help make valid predictions in ways that help exert power over our
environments.

Knowledge of Neurons

\pandocbounded{\includegraphics[keepaspectratio]{lecture_images/1L_prosthetic.mp4}}

Knowledge of Addiction

\includegraphics[width=6.05208in,height=\textheight,keepaspectratio]{lecture_images/1L_fb.png}

\marginnote{\begin{footnotesize}

\href{https://www.nytimes.com/2016/05/06/business/facebook-bends-the-rules-of-audience-engagement-to-its-advantage.html}{Link
to NYT article}. These data are a little dated; couldn't find more
recent data on this, but in
\href{https://s21.q4cdn.com/399680738/files/doc_financials/2024/q2/Earnings-Presentation-Q2-2024.pdf}{reports
to investors} reports consistent growth in metrics like ``ad
impressions'' and ``daily active users''.

\end{footnotesize}}

\subsection{The Linear Model}\label{the-linear-model}

\begin{center}
\includegraphics[width=7.69792in,height=\textheight,keepaspectratio]{images/clipboard-606362939.png}
\end{center}

\subsubsection{RECAP : Definition and
Examples}\label{recap-definition-and-examples}

\begin{itemize}
\item
  DV \textasciitilde{} IV1 + IV2 + IV3 + \ldots{} + ERROR

  \begin{itemize}
  \item
    \textbf{list the variable that you want to predict (the DV)}
  \item
    \textbf{list the variables that you think will help predict the DV
    (the IVs)}

    \begin{itemize}
    \item
      \textbf{NOTE : you do not (and can not) account for EVERY variable
      in your linear model!}
    \item
      \textbf{IN FACT : researchers are very specific about the
      variables they will include.}
    \item
      \textbf{ERROR will capture all the other variables not in your
      model.}
    \end{itemize}
  \end{itemize}
\item
  Example : Define a linear model to predict why people differ in the
  amount of screen time that they use (between or within-person
  differences)?
\end{itemize}

\subsubsection{KEY IDEA : Linear Models Help Organize and Quantify
Prediction}\label{key-idea-linear-models-help-organize-and-quantify-prediction}

\begin{itemize}
\tightlist
\item
  \textbf{organize :} a model specifies what information (IV) we think
  might help us predict the DV
\item
  \textbf{quantify :} when we add statistics to our model (much later!),
  we will see\ldots{}

  \begin{itemize}
  \item
    \textbf{the direction of the prediction :} is there a positive or
    negative relationship between the IV and the DV?
  \item
    \textbf{the ``weight'' of the prediction :} how much does each IV
    help us predict the DV?

    \begin{itemize}
    \item
      what is the ``best'' predictor in our model?
    \item
      what is the ``worst'' predictor in our model?
    \end{itemize}
  \item
    \textbf{the amount of error in our prediction :} how well does the
    model as a whole help us predict individual scores? how well does
    the model generalize to ``reality''?
  \end{itemize}
\end{itemize}

\subsubsection{A Real-Life Linear Model in the
Wild.}\label{a-real-life-linear-model-in-the-wild.}

\pandocbounded{\includegraphics[keepaspectratio]{images/clipboard-1655892865.png}}

\subsubsection{From Prediction to Linear
Models}\label{from-prediction-to-linear-models}

\begin{itemize}
\tightlist
\item
  I predict that attendance will improve students grades in the class.
\end{itemize}

\texttt{\textasciitilde{}\ attendance\ \textasciitilde{}\ grades\ +\ error}

\begin{itemize}
\tightlist
\item
  Write a model for your prediction (Lab 1, Q2)
\end{itemize}

\subsection{Is science adequate?}\label{is-science-adequate}

``The revolutionary viewpoint of a movement which thinks it can dominate
current history by means of scientific knowledge remains
\emph{bourgeois\ldots{}}the utopian socialists, remaining prisoners of
the mode of exposition of scientific truth, conceived this truth in
terms of its pure abstract image\ldots.it is on the model of
\emph{astronomy} that the utopians thought they would discover and
demonstrate the laws of society.'' (82-83)

\begin{itemize}
\tightlist
\item
  What makes sense / is confusing about this quote?
\item
  What are some problems with ``the model of astronomy''?
\end{itemize}

\pandocbounded{\includegraphics[keepaspectratio]{images/clipboard-1625483785.png}}

\subsection{Is science adequate?}\label{is-science-adequate-1}

``If you want knowledge, you must take part in the practice of changing
reality. If you want to know the taste of a pear, you must change the
pear by eating it yourself. If you want to know the structure and
properties of the atom, you must make physical and chemical experiments
to change the state of the atom. If you want to know the theory and
methods of revolution, you must take part in revolution. All genuine
knowledge originates in direct experience.'' -
\href{https://www.marxists.org/reference/archive/mao/selected-works/volume-1/mswv1_16.htm}{Mao}

\href{https://pacscenter.stanford.edu/publication/extreme-protest-tactics-reduce-popular-support-for-social-movements/}{\begin{center}
\pandocbounded{\includegraphics[keepaspectratio]{images/clipboard-3026795340.png}}
\end{center}
}

\subsubsection{Take this Exit Survey Now:
tinyurl.com/onlinebyeintro}\label{take-this-exit-survey-now-tinyurl.comonlinebyeintro}

\pandocbounded{\includegraphics[keepaspectratio]{1L_WhyStats_files/mediabag/frog-dance-frog.gif}}

\section{Next Time on Psych 101.}\label{next-time-on-psych-101.}

\begin{itemize}
\tightlist
\item
  Attend Discussion Section \& Complete Lab 1
\item
  Read Chapter 2 \& take Quiz 2
\end{itemize}

\section{So You Think You Want to Be a
Researcher?}\label{so-you-think-you-want-to-be-a-researcher}

\subsection{Getting Research Experience as an
RA}\label{getting-research-experience-as-an-ra}

\subsubsection{\texorpdfstring{\textbf{RA = Research
Assistant}}{RA = Research Assistant}}\label{ra-research-assistant}

\begin{itemize}
\item
  Mostly Unpaid Experiences

  \begin{itemize}
  \item
    From the berkeley website\ldots{}
  \item
    ``Cold calling'' labs who are doing work you think is cool.
  \item
    Chat with your TAs / Professors
  \end{itemize}
\item
  Some paid experiences exist!

  \begin{itemize}
  \item
    \href{https://haas.berkeley.edu/mors/faculty/}{Busine\$\$ \$chool}
  \item
    \href{https://docs.google.com/spreadsheets/d/1IX7f1N_VXUkSnnJPTpLg-pUto0427XBK_jxRxNFhmps/edit?gid=1345132026\#gid=1345132026}{A
    list a student sent me that they found}.
  \end{itemize}
\end{itemize}

\subsubsection{Work You Do as an RA}\label{work-you-do-as-an-ra}

\begin{itemize}
\item
  \textbf{work with data :} transcribing data; behavioral coding data;
  recruiting and participants to collect data; setting up
  psychophysiological recordings; cleaning data; etc.
\item
  \textbf{other opportunities to gain skills you can demonstrate :}

  \begin{itemize}
  \item
    reading \& discussing papers
  \item
    working with IRB (institutional review board - an ethics thing)
  \item
    analyzing data → presenting research at a conference (poster) or
    submitting a paper for publication {[}your golden ticket{]}
  \item
    general mentorship (how to apply to grad school; where to apply; who
    to talk to \& e-mail; etc.)
  \item
    NOTE : this work and these skills apply to other work outside of
    research applications {[}time management; coordinating schedules;
    juggling responsibilities; etc.{]}
  \end{itemize}
\end{itemize}

\subsubsection{What You Get Out of This}\label{what-you-get-out-of-this}

\begin{itemize}
\item
  \textbf{Course credit (hah), a letter of recommendation, ability to
  write about experiences that you have had (see above).}
\item
  \textbf{Vibes : is this {[}work or lab{]} for you?}

  \begin{itemize}
  \item
    do you enjoy the work? are you going to look forward to showing up
    and doing the work / fulfilling the commitment?~
  \item
    are you working with a horrible monster?

    \begin{itemize}
    \item
      not responsive
    \item
      inconsistent work / no plan for your work
    \item
      kind of a bully (emotionally abusive → stealing your work)~
    \end{itemize}
  \item
    or are you working with someone who is super cool and a positive
    influence on mentoring young minds!?!?! {[}YES!!!!{]}
  \end{itemize}
\end{itemize}

\subsection{Applying to Graduate
School}\label{applying-to-graduate-school}

\begin{itemize}
\item
  \textbf{You are applying to work on research with a specific
  professor(s) at a school.}

  \begin{itemize}
  \item
    Should have a sense of the topic you want to pursue.
  \item
    Good to have a narrative about how your past work and studies have
    prepared you for this topic / demonstrate an enduring interest in
    the topic.
  \end{itemize}
\item
  \textbf{Independent Thesis / Research Project :}

  \begin{itemize}
  \item
    \textbf{an official honors' thesis}
  \item
    \textbf{undergraduate research project (e.g., SURF; Psych 101!)}
  \item
    \textbf{your own independent study / advanced work you did as an RA}
  \end{itemize}
\item
  \textbf{Personal Statement : Experiences with Research You Can Write
  About}

  \begin{itemize}
  \item
    \st{I'm fascinated by people\ldots Over the last year, I worked on
    an independent research study to better understand\ldots.}
  \item
    Working as an RA; your research project; attending / presenting at a
    conference; etc.
  \end{itemize}
\item
  \textbf{3-4 Letters of Recommendation :} folks who can speak
  personally to your ability to do research.
\item
  \textbf{Clinical Students : some kind of clinical internship /
  experience 😟}
\item
  \textbf{Talk to people who are doing the thing you want to be doing
  about their journey}
\end{itemize}

\subsubsection{The Academic Job Market}\label{the-academic-job-market}

\begin{longtable}[]{@{}
  >{\raggedright\arraybackslash}p{(\linewidth - 2\tabcolsep) * \real{0.1500}}
  >{\raggedright\arraybackslash}p{(\linewidth - 2\tabcolsep) * \real{0.8500}}@{}}
\caption{Academic Job Market Data
{[}\href{https://ncses.nsf.gov/pubs/nsf24300/report/postgraduation-trends\#definite-commitments-at-graduation}{Source}{]}}\tabularnewline
\toprule\noalign{}
\endfirsthead
\endhead
\bottomrule\noalign{}
\endlastfoot
PhDs get jobs? &
\pandocbounded{\includegraphics[keepaspectratio]{lecture_images/1L_jobs1.png}} \\
but not in academia\ldots{} &
\pandocbounded{\includegraphics[keepaspectratio]{lecture_images/1L_jobs2.png}} \\
\$\$\$\$\$\$\$\$ &
\pandocbounded{\includegraphics[keepaspectratio]{lecture_images/1L_jobs3.png}} \\
\end{longtable}




\end{document}
