% Options for packages loaded elsewhere
% Options for packages loaded elsewhere
\PassOptionsToPackage{unicode}{hyperref}
\PassOptionsToPackage{hyphens}{url}
\PassOptionsToPackage{dvipsnames,svgnames,x11names}{xcolor}
%
\documentclass[
  letterpaper,
  DIV=11,
  numbers=noendperiod]{scrartcl}
\usepackage{xcolor}
\usepackage{amsmath,amssymb}
\setcounter{secnumdepth}{-\maxdimen} % remove section numbering
\usepackage{iftex}
\ifPDFTeX
  \usepackage[T1]{fontenc}
  \usepackage[utf8]{inputenc}
  \usepackage{textcomp} % provide euro and other symbols
\else % if luatex or xetex
  \usepackage{unicode-math} % this also loads fontspec
  \defaultfontfeatures{Scale=MatchLowercase}
  \defaultfontfeatures[\rmfamily]{Ligatures=TeX,Scale=1}
\fi
\usepackage{lmodern}
\ifPDFTeX\else
  % xetex/luatex font selection
\fi
% Use upquote if available, for straight quotes in verbatim environments
\IfFileExists{upquote.sty}{\usepackage{upquote}}{}
\IfFileExists{microtype.sty}{% use microtype if available
  \usepackage[]{microtype}
  \UseMicrotypeSet[protrusion]{basicmath} % disable protrusion for tt fonts
}{}
\makeatletter
\@ifundefined{KOMAClassName}{% if non-KOMA class
  \IfFileExists{parskip.sty}{%
    \usepackage{parskip}
  }{% else
    \setlength{\parindent}{0pt}
    \setlength{\parskip}{6pt plus 2pt minus 1pt}}
}{% if KOMA class
  \KOMAoptions{parskip=half}}
\makeatother
% Make \paragraph and \subparagraph free-standing
\makeatletter
\ifx\paragraph\undefined\else
  \let\oldparagraph\paragraph
  \renewcommand{\paragraph}{
    \@ifstar
      \xxxParagraphStar
      \xxxParagraphNoStar
  }
  \newcommand{\xxxParagraphStar}[1]{\oldparagraph*{#1}\mbox{}}
  \newcommand{\xxxParagraphNoStar}[1]{\oldparagraph{#1}\mbox{}}
\fi
\ifx\subparagraph\undefined\else
  \let\oldsubparagraph\subparagraph
  \renewcommand{\subparagraph}{
    \@ifstar
      \xxxSubParagraphStar
      \xxxSubParagraphNoStar
  }
  \newcommand{\xxxSubParagraphStar}[1]{\oldsubparagraph*{#1}\mbox{}}
  \newcommand{\xxxSubParagraphNoStar}[1]{\oldsubparagraph{#1}\mbox{}}
\fi
\makeatother

\usepackage{color}
\usepackage{fancyvrb}
\newcommand{\VerbBar}{|}
\newcommand{\VERB}{\Verb[commandchars=\\\{\}]}
\DefineVerbatimEnvironment{Highlighting}{Verbatim}{commandchars=\\\{\}}
% Add ',fontsize=\small' for more characters per line
\usepackage{framed}
\definecolor{shadecolor}{RGB}{241,243,245}
\newenvironment{Shaded}{\begin{snugshade}}{\end{snugshade}}
\newcommand{\AlertTok}[1]{\textcolor[rgb]{0.68,0.00,0.00}{#1}}
\newcommand{\AnnotationTok}[1]{\textcolor[rgb]{0.37,0.37,0.37}{#1}}
\newcommand{\AttributeTok}[1]{\textcolor[rgb]{0.40,0.45,0.13}{#1}}
\newcommand{\BaseNTok}[1]{\textcolor[rgb]{0.68,0.00,0.00}{#1}}
\newcommand{\BuiltInTok}[1]{\textcolor[rgb]{0.00,0.23,0.31}{#1}}
\newcommand{\CharTok}[1]{\textcolor[rgb]{0.13,0.47,0.30}{#1}}
\newcommand{\CommentTok}[1]{\textcolor[rgb]{0.37,0.37,0.37}{#1}}
\newcommand{\CommentVarTok}[1]{\textcolor[rgb]{0.37,0.37,0.37}{\textit{#1}}}
\newcommand{\ConstantTok}[1]{\textcolor[rgb]{0.56,0.35,0.01}{#1}}
\newcommand{\ControlFlowTok}[1]{\textcolor[rgb]{0.00,0.23,0.31}{\textbf{#1}}}
\newcommand{\DataTypeTok}[1]{\textcolor[rgb]{0.68,0.00,0.00}{#1}}
\newcommand{\DecValTok}[1]{\textcolor[rgb]{0.68,0.00,0.00}{#1}}
\newcommand{\DocumentationTok}[1]{\textcolor[rgb]{0.37,0.37,0.37}{\textit{#1}}}
\newcommand{\ErrorTok}[1]{\textcolor[rgb]{0.68,0.00,0.00}{#1}}
\newcommand{\ExtensionTok}[1]{\textcolor[rgb]{0.00,0.23,0.31}{#1}}
\newcommand{\FloatTok}[1]{\textcolor[rgb]{0.68,0.00,0.00}{#1}}
\newcommand{\FunctionTok}[1]{\textcolor[rgb]{0.28,0.35,0.67}{#1}}
\newcommand{\ImportTok}[1]{\textcolor[rgb]{0.00,0.46,0.62}{#1}}
\newcommand{\InformationTok}[1]{\textcolor[rgb]{0.37,0.37,0.37}{#1}}
\newcommand{\KeywordTok}[1]{\textcolor[rgb]{0.00,0.23,0.31}{\textbf{#1}}}
\newcommand{\NormalTok}[1]{\textcolor[rgb]{0.00,0.23,0.31}{#1}}
\newcommand{\OperatorTok}[1]{\textcolor[rgb]{0.37,0.37,0.37}{#1}}
\newcommand{\OtherTok}[1]{\textcolor[rgb]{0.00,0.23,0.31}{#1}}
\newcommand{\PreprocessorTok}[1]{\textcolor[rgb]{0.68,0.00,0.00}{#1}}
\newcommand{\RegionMarkerTok}[1]{\textcolor[rgb]{0.00,0.23,0.31}{#1}}
\newcommand{\SpecialCharTok}[1]{\textcolor[rgb]{0.37,0.37,0.37}{#1}}
\newcommand{\SpecialStringTok}[1]{\textcolor[rgb]{0.13,0.47,0.30}{#1}}
\newcommand{\StringTok}[1]{\textcolor[rgb]{0.13,0.47,0.30}{#1}}
\newcommand{\VariableTok}[1]{\textcolor[rgb]{0.07,0.07,0.07}{#1}}
\newcommand{\VerbatimStringTok}[1]{\textcolor[rgb]{0.13,0.47,0.30}{#1}}
\newcommand{\WarningTok}[1]{\textcolor[rgb]{0.37,0.37,0.37}{\textit{#1}}}

\usepackage{longtable,booktabs,array}
\usepackage{calc} % for calculating minipage widths
% Correct order of tables after \paragraph or \subparagraph
\usepackage{etoolbox}
\makeatletter
\patchcmd\longtable{\par}{\if@noskipsec\mbox{}\fi\par}{}{}
\makeatother
% Allow footnotes in longtable head/foot
\IfFileExists{footnotehyper.sty}{\usepackage{footnotehyper}}{\usepackage{footnote}}
\makesavenoteenv{longtable}
\usepackage{graphicx}
\makeatletter
\newsavebox\pandoc@box
\newcommand*\pandocbounded[1]{% scales image to fit in text height/width
  \sbox\pandoc@box{#1}%
  \Gscale@div\@tempa{\textheight}{\dimexpr\ht\pandoc@box+\dp\pandoc@box\relax}%
  \Gscale@div\@tempb{\linewidth}{\wd\pandoc@box}%
  \ifdim\@tempb\p@<\@tempa\p@\let\@tempa\@tempb\fi% select the smaller of both
  \ifdim\@tempa\p@<\p@\scalebox{\@tempa}{\usebox\pandoc@box}%
  \else\usebox{\pandoc@box}%
  \fi%
}
% Set default figure placement to htbp
\def\fps@figure{htbp}
\makeatother

\ifLuaTeX
  \usepackage{luacolor}
  \usepackage[soul]{lua-ul}
\else
  \usepackage{soul}
\fi




\setlength{\emergencystretch}{3em} % prevent overfull lines

\providecommand{\tightlist}{%
  \setlength{\itemsep}{0pt}\setlength{\parskip}{0pt}}



 


\KOMAoption{captions}{tableheading}
\makeatletter
\@ifpackageloaded{caption}{}{\usepackage{caption}}
\AtBeginDocument{%
\ifdefined\contentsname
  \renewcommand*\contentsname{Table of contents}
\else
  \newcommand\contentsname{Table of contents}
\fi
\ifdefined\listfigurename
  \renewcommand*\listfigurename{List of Figures}
\else
  \newcommand\listfigurename{List of Figures}
\fi
\ifdefined\listtablename
  \renewcommand*\listtablename{List of Tables}
\else
  \newcommand\listtablename{List of Tables}
\fi
\ifdefined\figurename
  \renewcommand*\figurename{Figure}
\else
  \newcommand\figurename{Figure}
\fi
\ifdefined\tablename
  \renewcommand*\tablename{Table}
\else
  \newcommand\tablename{Table}
\fi
}
\@ifpackageloaded{float}{}{\usepackage{float}}
\floatstyle{ruled}
\@ifundefined{c@chapter}{\newfloat{codelisting}{h}{lop}}{\newfloat{codelisting}{h}{lop}[chapter]}
\floatname{codelisting}{Listing}
\newcommand*\listoflistings{\listof{codelisting}{List of Listings}}
\makeatother
\makeatletter
\makeatother
\makeatletter
\@ifpackageloaded{caption}{}{\usepackage{caption}}
\@ifpackageloaded{subcaption}{}{\usepackage{subcaption}}
\makeatother
\usepackage{bookmark}
\IfFileExists{xurl.sty}{\usepackage{xurl}}{} % add URL line breaks if available
\urlstyle{same}
\hypersetup{
  pdftitle={Class 1 \textbar{} Why Statistics?},
  colorlinks=true,
  linkcolor={blue},
  filecolor={Maroon},
  citecolor={Blue},
  urlcolor={Blue},
  pdfcreator={LaTeX via pandoc}}


\title{Class 1 \textbar{} Why Statistics?}
\author{}
\date{}
\begin{document}
\maketitle


\section{Hello Again!}\label{hello-again}

\href{https://docs.google.com/forms/d/e/1FAIpQLSekU91xhyXiYxxOsWbSGuJ3pAmPOazq9L_24hkcQpej-n0bBg/viewform?usp=header}{\textbf{click
on this link to check-in (or visit : tinyurl.com/first101class)}}

\subsection{Announcements}\label{announcements}

\begin{itemize}
\tightlist
\item
  \textbf{Section Swap :} post on bCourses to find someone to swap with.
\item
  \textbf{Waitlisted Students :} Thanks for your patience! Go
  bears\ldots{}
\item
  \textbf{Join the Class Discord :} link on bCourses
\item
  \textbf{Next Week :}

  \begin{itemize}
  \tightlist
  \item
    Attend Discussion Section
  \item
    Read Chapter 1
  \item
    Complete Quiz 1
  \end{itemize}
\end{itemize}

\subsection{Agenda}\label{agenda}

\begin{itemize}
\tightlist
\item
  2:10 - 2:20 : Check-In \& Announcements
\item
  2:20 - 2:30 : RECAP : Variables and Variation
\item
  2:30 - 3:20 : IN R : Defining Variables
\item
  4:20 - 4:30 : Break \#1
\item
  4:30 - 5:00 : RECAP : Prediction and Linear Models
\item
  5:00 - 5:05 : Break \#2
\item
  5:05 - 5:30 : DISCUSSION : Research Questions as Linear Model
\item
  5:30 - 6:00 : So you're interested in going to graduate school?
\end{itemize}

\section{Part 1 : Who and Why and What
Statistics?}\label{part-1-who-and-why-and-what-statistics}

\subsection{Who Statistics?}\label{who-statistics}

\begin{itemize}
\item
  \textbf{Professor :} Arman (Daniel) Catterson

  \begin{itemize}
  \item
    \textbf{say :} Arman\ldots Professor\ldots Professor
    Catterson\ldots Dr.~Catterson
  \item
    \textbf{from :} Austin, TX to UC Berkeley to stayin in the bay
    forever 😎 teaching here and at Diablo Valley College
  \item
  \end{itemize}
\end{itemize}

\subsection{Why Statistics?}\label{why-statistics}

\subsection{What Statistics?}\label{what-statistics}

\subsection{Activity : Variables and Variation in the
Room}\label{activity-variables-and-variation-in-the-room}

\textbf{Class Activity.} Let's create a list of variables that we
observe in this classroom.

\begin{itemize}
\item
  color of shirt
\item
  glasses
\item
  hair length
\item
  gender

  \begin{itemize}
  \item
    affect : gender norms in how we are raised to process emotions
    (example : boys are taught to suppress SOME emotions, and okay /
    encouraged to express others like AnGER, whereas girls are
    perceieved as being ``overly-emotional''; but okay to be
    ANGRY\ldots.''
  \item
    behavior : ``gender norms'' in terms of roles, occupations,
    behavior, language (e.g., women are ``homemakers'' who raise the
    children.)
  \item
    cognition : how you perceive yourself based on your gender
    (transgender identity); how you perceive yourself acting in a gender
    role in relationship that is in a patriarchal context (OR NOT!)
  \end{itemize}
\item
  transfer vs/ non-transfer
\item
  eye color
\item
  height
\end{itemize}

\textbf{Key Terms.} From the readings.

\begin{itemize}
\item
  Affect, Behavior, Cognition
\item
  Between vs.~Within-Person Variation
\end{itemize}

\textbf{{[}7 Minutes{]} Answer the following questions with your buddy.}

Find a buddy in the class! (There's a discord thread if you prefer to
communicate with someone online.)

\begin{itemize}
\item
  If you could have dinner with anyone in the world (living or dead) who
  would it be?
\item
  Why are you a psychology major? What interests you about people (or
  non-human animals)?
\item
  How would you label this interest as a variable?

  \begin{itemize}
  \item
    Are you interested in the between-person or within-person version of
    this variable?
  \item
    Are you interested in the Affective, Behavioral, or Cognitive aspect
    of this variable?
  \end{itemize}
\end{itemize}

\textbf{Student Examples}

\begin{itemize}
\item
  liam : living on vs.~off campus and how that perceived stress

  \begin{itemize}
  \item
    stress : affect
  \item
    living situation (on vs.~off) : behavior
  \end{itemize}
\item
  isabella : disorders (psychosis or schizophrenia)
\end{itemize}

\textbf{Student Questions :}

\begin{itemize}
\item
  is it possible to study how people become emotionally involved with
  AI?

  \begin{itemize}
  \tightlist
  \item
    YES!!!
  \end{itemize}
\end{itemize}

\section{IN R : Defining Variables}\label{in-r-defining-variables}

\subsection{Thinking about Programming (Free Association
Activity)}\label{thinking-about-programming-free-association-activity}

\begin{itemize}
\tightlist
\item
  Close your eyes
\item
  Take a deep breath (inhale / exhale)
\item
  Visualize an image based on the word that you hear me say.
\item
  What do you observe?
\end{itemize}

\subsection{The R Console}\label{the-r-console}

The console is where R does its work.

\begin{itemize}
\item
  \textbf{ACTIVITY :} Look at the image below. What do you see? What
  makes sense / what seems confusing?
\item
  \includegraphics[width=5.4375in,height=\textheight,keepaspectratio]{lecture_images/1_RConsole.png}
\end{itemize}

\subsection{R Studio and Source Files
(.R)}\label{r-studio-and-source-files-.r}

In this class, we'll be using RStudio. RStudio is an IDE (Integrated
Development Environment) that includes the console along with other
useful windows and tools.

\begin{itemize}
\tightlist
\item
  \textbf{The Console} is at the bottom left of the IDE. Hi console!
\item
  \textbf{The R script} is at the top left of the IDE, and is a document
  that you use to write (and organize) code. You will want to do most of
  your work in the R script, and feel an appropriate level of anxiety
  when you notice that your Rscript is unsaved (as indicated by the red
  text and *).
\item
  \textbf{The Environment} is at the top right of the IDE, and shows you
  all of the ``objects'' that you have defined in R.
\item
  \textbf{The File Window} is at the bottom right of the IDE, and shows
  you the files. Note that there are tabs here for Plots (where graphs
  will pop up), Packages (things you can download to give R extra
  features), a Help viewer (sometimes very useful!).
\end{itemize}

\pandocbounded{\includegraphics[keepaspectratio]{lecture_images/clipboard-1807596003.png}}

\subsubsection{\texorpdfstring{\textbf{ACTIVTY :} open up
RStudio}{ACTIVTY : open up RStudio}}\label{activty-open-up-rstudio}

\begin{enumerate}
\def\labelenumi{\arabic{enumi}.}
\tightlist
\item
  \textbf{You are Programming!} Type some math into an Rscript, and send
  it to the console. THAT'S RIGHT.
\item
  \textbf{Work on Lab 1, Question 1 :} Define two variables in R : one
  numeric, and one string. \emph{Note : you can use the data that we
  collected in class, or collect your own data (make sure each variable
  has at least ten data points).} ``Print'' each variable in R, and
  paste the output as a screenshot. Then, graph the numeric variable as
  a ``histogram'' and the string variable as a ``plot''. Below your
  graph, describe what you observe about the individuals in the dataset
  for each variable. See the lecture videos in Chapter 2 for a guide on
  how to do this!
\end{enumerate}

\subsection{R CODE : Variables and
Variation}\label{r-code-variables-and-variation}

\subsubsection{Numeric Variables in R}\label{numeric-variables-in-r}

\begin{longtable}[]{@{}
  >{\raggedright\arraybackslash}p{(\linewidth - 2\tabcolsep) * \real{0.2519}}
  >{\raggedright\arraybackslash}p{(\linewidth - 2\tabcolsep) * \real{0.7481}}@{}}
\toprule\noalign{}
\begin{minipage}[b]{\linewidth}\raggedright
Code
\end{minipage} & \begin{minipage}[b]{\linewidth}\raggedright
Description
\end{minipage} \\
\midrule\noalign{}
\endhead
\bottomrule\noalign{}
\endlastfoot
\begin{minipage}[t]{\linewidth}\raggedright
\begin{Shaded}
\begin{Highlighting}[]
\NormalTok{variable }\OtherTok{\textless{}{-}} \FunctionTok{c}\NormalTok{(}\CommentTok{\#, \#, \#, \#, etc.)}



\NormalTok{tired }\OtherTok{\textless{}{-}} \FunctionTok{c}\NormalTok{(}\DecValTok{1}\NormalTok{,}\DecValTok{2}\NormalTok{,}\DecValTok{3}\NormalTok{,}\DecValTok{4}\NormalTok{)}
\end{Highlighting}
\end{Shaded}
\end{minipage} & \textbf{variable} = an object that you will define in R

\textbf{\textless-} = ``assign''; tells R to save whatever comes on the
right to whatever object is on the left.

\textbf{c} = combine : tells R to combine whatever happens in the
parentheses

\textbf{()} = parentheses to group related terms

\textbf{\# =} what you store in the variable; each item should be
separated by a comma and space. \\
\begin{minipage}[t]{\linewidth}\raggedright
\begin{Shaded}
\begin{Highlighting}[]
\FunctionTok{hist}\NormalTok{(dat}\SpecialCharTok{$}\NormalTok{variable)}
\end{Highlighting}
\end{Shaded}
\end{minipage} & \textbf{For continuous variables :} draws a
histogram. \\
\end{longtable}

\subsection{Example : Creating Numeric
Variables}\label{example-creating-numeric-variables}

Note : the code below might not run depending on your browser settings.
professor will demonstrate this in class using R.

\begin{Shaded}
\begin{Highlighting}[]
\NormalTok{prof.sleep \textless{}{-} c(5.5, 5, 7, 8.5, 6.5, 5, 7.0, 7.5, 7.5, 6, 6.5, 5) \# hours of sleep professor got the last few nights.}
\NormalTok{print(prof.sleep) \# one way to "print" the variable}
\NormalTok{prof.sleep \# another way to "print" the variable}
\NormalTok{hist(prof.sleep) \# a way to graph the variable (a histogram)}
\end{Highlighting}
\end{Shaded}

\subsection{String Variables}\label{string-variables}

\begin{longtable}[]{@{}
  >{\raggedright\arraybackslash}p{(\linewidth - 2\tabcolsep) * \real{0.2322}}
  >{\raggedright\arraybackslash}p{(\linewidth - 2\tabcolsep) * \real{0.7678}}@{}}
\toprule\noalign{}
\endhead
\bottomrule\noalign{}
\endlastfoot
\begin{minipage}[t]{\linewidth}\raggedright
\begin{Shaded}
\begin{Highlighting}[]
\NormalTok{variable }\OtherTok{\textless{}{-}} \FunctionTok{c}\NormalTok{(“name1”, “name2”, “name1”, etc.)}

\NormalTok{emotion }\OtherTok{\textless{}{-}} \FunctionTok{c}\NormalTok{(“sad”, “happy”, “sad”)}
\end{Highlighting}
\end{Shaded}
\end{minipage} & \textbf{variable} = an object that you will define in R

\textbf{\textless-} = ``assign''; tells R to save whatever comes on the
right to whatever object is on the left.

\textbf{c} = combine : tells R to combine whatever happens in the
parentheses

\textbf{()} = parentheses to group related terms

\textbf{\# =} what you store in the variable; each item should be
separated by a comma and space. \\
\begin{minipage}[t]{\linewidth}\raggedright
\begin{Shaded}
\begin{Highlighting}[]
\FunctionTok{as.factor}\NormalTok{(variable)}

\FunctionTok{as.factor}\NormalTok{(emotion)}
\end{Highlighting}
\end{Shaded}
\end{minipage} & \textbf{as.factor()} \# converts a string variable into
a categorical factor \\
\begin{minipage}[t]{\linewidth}\raggedright
\begin{verbatim}
variable <- as.factor(variable)
\end{verbatim}
\end{minipage} & \textbf{\# ``}saves'' this conversion as the original
variable \\
\begin{minipage}[t]{\linewidth}\raggedright
\begin{Shaded}
\begin{Highlighting}[]
\FunctionTok{plot}\NormalTok{(dat}\SpecialCharTok{$}\NormalTok{variable)}
\end{Highlighting}
\end{Shaded}
\end{minipage} & \textbf{For categorical variables :} draws a barplot.
For continuous variables :~ illustrates values of the variable (y-axis)
as a function of their index (x-axis). \\
\end{longtable}

\subsubsection{Example in R : Creating Non-Numeric
Variables}\label{example-in-r-creating-non-numeric-variables}

The data below describe the categories of family laundry that was
hanging in my apartment to dry. As before, the code below might not run
depending on your browser settings. professor will demonstrate this in
class using R.

\begin{Shaded}
\begin{Highlighting}[]
\NormalTok{laundryhang }\OtherTok{\textless{}{-}} \FunctionTok{c}\NormalTok{(}\StringTok{"shirt"}\NormalTok{, }\StringTok{"shirt"}\NormalTok{, }\StringTok{"leggings"}\NormalTok{, }\StringTok{"leggings"}\NormalTok{, }\StringTok{"shirt"}\NormalTok{, }
             \StringTok{"shirt"}\NormalTok{, }\StringTok{"leggings"}\NormalTok{, }\StringTok{"pants"}\NormalTok{, }\StringTok{"sweater"}\NormalTok{, }\StringTok{"sweater"}\NormalTok{) }\CommentTok{\# defining a string variable}
\FunctionTok{print}\NormalTok{(laundryhang)}
\end{Highlighting}
\end{Shaded}

\begin{verbatim}
 [1] "shirt"    "shirt"    "leggings" "leggings" "shirt"    "shirt"   
 [7] "leggings" "pants"    "sweater"  "sweater" 
\end{verbatim}

\begin{Shaded}
\begin{Highlighting}[]
\NormalTok{laundryhang }\CommentTok{\# another way to "print" the variable}
\end{Highlighting}
\end{Shaded}

\begin{verbatim}
 [1] "shirt"    "shirt"    "leggings" "leggings" "shirt"    "shirt"   
 [7] "leggings" "pants"    "sweater"  "sweater" 
\end{verbatim}

\begin{Shaded}
\begin{Highlighting}[]
\NormalTok{laundryhang }\OtherTok{\textless{}{-}} \FunctionTok{as.factor}\NormalTok{(laundryhang) }\CommentTok{\# changing the format of the sting variable into a categorical factor}
\FunctionTok{plot}\NormalTok{(laundryhang) }\CommentTok{\# a way to graph the non{-}numeric variable}
\end{Highlighting}
\end{Shaded}

\pandocbounded{\includegraphics[keepaspectratio]{1L_WhyStats_files/figure-pdf/unnamed-chunk-2-1.pdf}}

\subsection{Break Time: Meet Back at
4:25}\label{break-time-meet-back-at-425}

\subsection{Prediction \& Power}\label{prediction-power}

\subsubsection{Predictions in Real Life}\label{predictions-in-real-life}

\begin{itemize}
\tightlist
\item
  EXAMPLE : PROFESSOR WILL LET US OUT EARLY ON THE FIRST DAY

  \begin{itemize}
  \tightlist
  \item
    \textbf{Knowledge (what information did you use to make the
    prediction)?}

    \begin{itemize}
    \tightlist
    \item
      past experience with other professors
    \item
      syllabus seemed FUN and LIGHT HEARTED like someone who would let
      us out EARLY
    \item
      you want to get out early\ldots and so will MANIFEST that through
      the power of PREDICTION.
    \end{itemize}
  \item
    \textbf{Power (ways your predictions influence future behaviors)}

    \begin{itemize}
    \tightlist
    \item
      making a plan to leave
    \item
      trying to see if we can leverage this activity into actually
      getting out early (\#MANIFEST)
    \item
      did not bring a snack
    \end{itemize}
  \item
    \textbf{Was The Predicton Valid?}

    \begin{itemize}
    \tightlist
    \item
      WE WILL FIND OUT.
    \end{itemize}
  \item
    \textbf{Work on Lab Question.} What's a prediction about people that
    you made today? What information did you use to make this
    prediction? How did (or could) you use this prediction to influence
    outcomes? Were you valid in your predictions? Finally, write a
    linear model that defines the prediction (and information that you
    used) as a formula (e.g., DV \textasciitilde{} IV1 + IV2 + \ldots{}
    + error).
  \item
    STUDENT EXAMPLE : Naomi

    \begin{itemize}
    \tightlist
    \item
      prediction : was gonna be late to class
    \item
      knowledge : have practice until 3:30 on Mondays and coaches be
      like that.
    \item
      power :

      \begin{itemize}
      \tightlist
      \item
        rushed after you got let out
      \item
        ran uphill
      \item
        asked friend (whoI predicted was gonna get here early) to save a
        seat.
      \end{itemize}
    \item
      valid : yes, I was 2-minutes late.
    \item
      model :

      \begin{itemize}
      \tightlist
      \item
        time to get to class \textasciitilde{} previous class time +
        coach personality + campus geography + own personal walking
        speed after tennis practice + weather + ERROR
      \end{itemize}
    \end{itemize}
  \end{itemize}
\end{itemize}

\subsubsection{Scientific Predictions}\label{scientific-predictions}

Psychological scientists seek to better understand variation, in order
to help make valid predictions in ways that help exert power over our
environments.

\begin{longtable}[]{@{}
  >{\raggedright\arraybackslash}p{(\linewidth - 2\tabcolsep) * \real{0.6049}}
  >{\raggedright\arraybackslash}p{(\linewidth - 2\tabcolsep) * \real{0.3951}}@{}}
\toprule\noalign{}
\begin{minipage}[b]{\linewidth}\raggedright
Topic
\end{minipage} & \begin{minipage}[b]{\linewidth}\raggedright
Other Questions We Might Ask?
\end{minipage} \\
\midrule\noalign{}
\endhead
\bottomrule\noalign{}
\endlastfoot
\pandocbounded{\includegraphics[keepaspectratio]{lecture_images/1L_prosthetic.mp4}}
& \\
\includegraphics[width=6.25in,height=\textheight,keepaspectratio]{lecture_images/1L_fb.png}\footnote{Here's
  \href{https://www.nytimes.com/2016/05/06/business/facebook-bends-the-rules-of-audience-engagement-to-its-advantage.html}{a
  link to the article} where this headline comes from. These data are a
  little dated, and I couldn't immediately find more recent data - my
  guess is Meta does not really want to advertise that people are using
  the product more and more. However, in
  \href{https://s21.q4cdn.com/399680738/files/doc_financials/2024/q2/Earnings-Presentation-Q2-2024.pdf}{reports
  to investors} reports consistent growth in metrics like ``ad
  impressions'' and ``daily active users''. Let me know if you find
  other sources to show how technology companies are capturing more and
  more of our attention!} & \\
\end{longtable}

\subsection{The Linear Model}\label{the-linear-model}

\begin{center}
\includegraphics[width=7.69792in,height=\textheight,keepaspectratio]{images/clipboard-606362939.png}
\end{center}

\subsubsection{RECAP : Definition and
Examples}\label{recap-definition-and-examples}

\begin{itemize}
\item
  Steps to Take :~

  \begin{itemize}
  \item
    \textbf{list the variable that you want to predict (the DV)}
  \item
    \textbf{list the variables that you think will help predict the DV
    (the IVs)}

    \begin{itemize}
    \item
      \textbf{NOTE : you do not (and can not) account for EVERY variable
      in your linear model!}
    \item
      \textbf{IN FACT : researchers are very specific about the
      variables they will include.}
    \item
      \textbf{ERROR will capture all the other variables not in your
      model.}
    \end{itemize}
  \end{itemize}
\item
  Example : What predicts why people differ in the amount of screen time
  that they use (between or within-person differences)?

  \begin{itemize}
  \tightlist
  \item
  \end{itemize}
\end{itemize}

\textbf{KEY IDEA : Linear Models Help Organize and Quantify Prediction}

\begin{itemize}
\tightlist
\item
  \textbf{organize :} a model specifies what information (IV) we think
  might help us predict the DV
\item
  \textbf{quantify :} when we add statistics to our model (much later!),
  we will see\ldots{}

  \begin{itemize}
  \item
    \textbf{the direction of the prediction :} is there a positive or
    negative relationship between the IV and the DV?
  \item
    \textbf{the ``weight'' of the prediction :} how much does each IV
    help us predict the DV?

    \begin{itemize}
    \item
      what is the ``best'' predictor in our model?
    \item
      what is the ``worst'' predictor in our model?
    \end{itemize}
  \item
    \textbf{the amount of error in our prediction :} how well does the
    model as a whole help us predict individual scores? how well does
    the model generalize to ``reality''?
  \end{itemize}
\item
  \textbf{A Real-Life Linear Model in the Wild :}
\end{itemize}

\begin{center}
\includegraphics[width=4.07292in,height=\textheight,keepaspectratio]{1L_WhyStats_files/mediabag/AD_4nXcXn1H6Tcp4pWwk.pdf}
\end{center}

\subsubsection{From Prediction --\textgreater{} Linear
Model}\label{from-prediction-linear-model}

\begin{itemize}
\item
  student example prediction :

  \begin{itemize}
  \tightlist
  \item
    as a model :
  \end{itemize}
\item
  student example prediction :

  \begin{itemize}
  \tightlist
  \item
    as a model :
  \end{itemize}
\end{itemize}

\subsection{Work on Lab 1 : Predictions \& Research
Questions.}\label{work-on-lab-1-predictions-research-questions.}

\subsubsection{On Your Own}\label{on-your-own}

\textbf{Lab 1, Question 2 :} What's a prediction about people that you
made today? What information did you use to make this prediction? How
did (or could) you use this prediction to influence outcomes? Were you
valid in your predictions? Finally, write a linear model that defines
the prediction (and information that you used) as a formula (e.g., DV
\textasciitilde{} IV1 + IV2 + \ldots{} + error).

\textbf{Lab 1, Question 3 :} Identify a research question that you might
be interested to study as a psychologist (this could be what you wrote
about in Lab 1, or something new.) Then, define the DV for this
question, and explain what interests you about this question and how
this variable might be understood as an example of affect, behavior,
and/or cognition. Next, explain what between-person and within-person
variation might look like for this variable. Finally, identify some
other variables that you think will predict this DV, and write out this
question (and your theory) as a linear model.

\textbf{Lab 1, Quesiton 4 (In Discussion Section).} With your discussion
section, define each of the six biases described in the Goldacre (2010)
reading on cognitive biases, and come up with an example from real-life.

\section{Next Time on Psych 101.}\label{next-time-on-psych-101.}

\begin{itemize}
\tightlist
\item
  Take this Exit Survey Now : tinyurl.com/onlinebyeintro
\item
  Attend Discussion Section \& Complete Lab 1
\item
  Read Chapter 2 \& take Quiz 2
\end{itemize}

\section{So You Think You Want to Be a
Researcher?}\label{so-you-think-you-want-to-be-a-researcher}

\subsection{Getting Research Experience as an
RA}\label{getting-research-experience-as-an-ra}

\begin{itemize}
\item
  \textbf{RA = Research Assistant}

  \begin{itemize}
  \item
    Mostly Unpaid Experiences
  \item
    Some paid experiences exist!

    \begin{itemize}
    \item
      From the berkeley website\ldots{}
    \item
      \href{https://haas.berkeley.edu/mors/faculty/}{Busine\$\$ \$chool}
    \item
      Stanford {[}maybe paid{]}
    \item
      \href{https://docs.google.com/spreadsheets/d/1IX7f1N_VXUkSnnJPTpLg-pUto0427XBK_jxRxNFhmps/edit?gid=1345132026\#gid=1345132026}{A
      list a student sent me that they found}.
    \item
      ``Cold calling'' labs who are doing work you think is cool.
    \item
      Chat with your TAs / Professors
    \end{itemize}
  \end{itemize}
\item
  \textbf{As an RA :}

  \begin{itemize}
  \item
    work with data : transcribing data; behavioral coding data;
    recruiting and participants to collect data; setting up
    psychophysiological recordings; cleaning data; etc.
  \item
    \textbf{other opportunities to gain skills you can demonstrate :}

    \begin{itemize}
    \item
      reading \& discussing papers
    \item
      working with IRB (institutional review board - an ethics thing)
    \item
      analyzing data → presenting research at a conference (poster) or
      submitting a paper for publication {[}your golden ticket{]}
    \item
      general mentorship (how to apply to grad school; where to apply;
      who to talk to \& e-mail; etc.)
    \item
      NOTE : this work and these skills apply to other work outside of
      research applications {[}time management; coordinating schedules;
      juggling responsibilities; etc.{]}
    \end{itemize}
  \item
    \textbf{get a sense of whether this {[}work or lab{]} is for you?}

    \begin{itemize}
    \item
      do you enjoy the work? are you going to look forward to showing up
      and doing the work / fulfilling the commitment?~
    \item
      are you working with a horrible monster?

      \begin{itemize}
      \item
        not responsive
      \item
        inconsistent work / no plan for your work
      \item
        kind of a bully (emotionally abusive → stealing your work)~
      \end{itemize}
    \item
      or are you working with someone who is super cool and a positive
      influence on mentoring young minds!?!?! {[}YES!!!!{]}
    \end{itemize}
  \end{itemize}
\end{itemize}

\subsection{Applying to Graduate
School}\label{applying-to-graduate-school}

\begin{itemize}
\item
  \textbf{You are applying to work on research with a specific
  professor(s) at a school.}

  \begin{itemize}
  \item
    Should have a sense of the topic you want to pursue.
  \item
    Good to have a narrative about how your past work and studies have
    prepared you for this topic / demonstrate an enduring interest in
    the topic.
  \end{itemize}
\item
  \textbf{Independent Thesis / Research Project :}

  \begin{itemize}
  \item
    \textbf{an official honors' thesis}
  \item
    \textbf{undergraduate research project (e.g., SURF; Psych 101!)}
  \item
    \textbf{your own independent study / advanced work you did as an RA}
  \end{itemize}
\item
  \textbf{Personal Statement : Experiences with Research You Can Write
  About}

  \begin{itemize}
  \item
    \st{I'm fascinated by people\ldots Over the last year, I worked on
    an independent research study to better understand\ldots.}
  \item
    Working as an RA; your research project; attending / presenting at a
    conference; etc.
  \end{itemize}
\item
  \textbf{3-4 Letters of Recommendation :} folks who can speak
  personally to your ability to do research.
\item
  \textbf{Clinical Students : some kind of clinical internship /
  experience 😟}
\item
  \textbf{Talk to people who are doing the thing you want to be doing
  about their journey}
\end{itemize}

\subsubsection{The Academic Job Market}\label{the-academic-job-market}

Some Data
{[}\href{https://ncses.nsf.gov/pubs/nsf24300/report/postgraduation-trends\#definite-commitments-at-graduation}{Source}{]}

\begin{longtable}[]{@{}
  >{\raggedright\arraybackslash}p{(\linewidth - 2\tabcolsep) * \real{0.3194}}
  >{\raggedright\arraybackslash}p{(\linewidth - 2\tabcolsep) * \real{0.6250}}@{}}
\toprule\noalign{}
\endhead
\bottomrule\noalign{}
\endlastfoot
PhDs get jobs? &
\pandocbounded{\includegraphics[keepaspectratio]{lecture_images/1L_jobs1.png}} \\
but not in academia\ldots{} &
\pandocbounded{\includegraphics[keepaspectratio]{lecture_images/1L_jobs2.png}} \\
\$\$\$\$\$\$\$\$ &
\pandocbounded{\includegraphics[keepaspectratio]{lecture_images/1L_jobs3.png}} \\
\end{longtable}




\end{document}
